% Einbindung von wichigten Paketen
\usepackage[headsepline=on, automark]{scrlayer-scrpage}
\usepackage[english, main=ngerman]{babel}
\usepackage[utf8]{inputenc}
\usepackage[T1]{fontenc}
\usepackage{lmodern}
\usepackage{microtype}
\usepackage{scrdate}
\usepackage{scrtime}
\usepackage{lastpage}
\usepackage{upgreek}
\usepackage[svgnames]{xcolor}
\usepackage{blindtext} % Erzeugt Blindtext. Nur f\"{u}r Testlauf dieses Dokuments wichtig.
\usepackage{graphicx}
\usepackage{siunitx}
\usepackage{booktabs}
\usepackage{longtable}
\usepackage{threeparttable}
\usepackage{typearea}
\usepackage{tikz}
\usepackage{tikzpagenodes}
\usepackage{epigraph}
\usepackage{ellipsis}
\usepackage{soul}
\usepackage[nohyperlinks]{acronym}
\usepackage[autostyle,german=guillemets]{csquotes}
\usepackage{eurosym}

% Die AMS-Mathe-Pakete
\usepackage{amsmath}
\usepackage{amsthm}
\usepackage{amsfonts}
%%%%%%%%%%%%%%%%%%%%%

\usepackage{setspace}
\usepackage{hyperref} % immer als letztes Paket laden
% Hintergrundvariablen des PDFs
\hypersetup{pdfauthor={Martina Musterfrau},
            pdftitle={Abschlussarbeit},
            pdfsubject={Abschlussarbeit},
            pdfkeywords={Abschlussarbeit PTRA},
            pdfproducer={Latex with hyperref},
            pdfcreator={pdflatex, MikTeX, WinEDT oder andere}
            }

% Farbdefinitionen (insbesonder die JLU-Farben werden hier definiert)
\definecolor{jluhellblau}{RGB}{220, 230, 235}
\definecolor{pantoneblack}{RGB}{0,0,0}
\definecolor{jlugrau}{RGB}{83, 96, 107}
\definecolor{jlublau}{RGB}{0, 105, 179}
\definecolor{chaptergrey}{rgb}{0.7,0.7,0.7}

% Kopf- und Fu{\ss}zeilenlayout
\clearpairofpagestyles
\automark[chapter]{chapter}
\automark*[section]{}
\ihead{}
\ohead{\headmark}
\ifoot{Erstellt am \ISOToday T\thistime} % f\"{u}r finales Dokument auskommentieren
%\ifoot{} % f\"{u}r finales Dokument aktivieren
\ofoot{\,\pagemark\ von \pageref*{LastPage}}
\chead{}
\cfoot[]{}

% Fu{\ss}notenmarker fett
\deffootnote{1em}{1em}{\bfseries\thefootnotemark\ }

% Formatierung Tabellen- und Abbildungsbeschriftung
\setcapindent{0pt} % keine Einr\"{u}ckung der Caption
\renewcommand{\floatpagefraction}{1.0}
\addtokomafont{caption}{\footnotesize}
\setkomafont{captionlabel}{\bfseries\sffamily}

% Literaturverzeichnis mit BibLaTeX
\usepackage[backend = biber, style=numeric-comp, sorting = none, giveninits = true]{biblatex}
\ifjournalabbrev
\usepackage{biblatex-shortfields}
\else\fi

\DefineBibliographyStrings{ngerman}{andothers = {{et\,al\adddot}},}
\addbibresource{mybibFile.bib}




% Es ist im Literaturverzeichnis unn\"{o}tig, DOI, URL und eprint gleichzeitig zu zeigen. Dieser Abschnitt macht eine
% Fallunterscheidung. Ist die DOI vorhanden, werden URL und eprint nicht angezeigt. Ist DOI nicht vorhanden, aber
% die beiden anderen Eintr\"{a}ge, wird die URL ausgegeben, nicht aber der eprint-Eintrag.
\renewbibmacro*{doi+eprint+url}{%
  \iftoggle{bbx:doi}
    {\printfield{doi}}
    {}%
  \newunit\newblock
  \ifboolexpr{togl {bbx:eprint} and test {\iffieldundef{doi}}}
    {\usebibmacro{eprint}}
    {}%
  \newunit\newblock
  \ifboolexpr{togl {bbx:url} and test {\iffieldundef{doi}}  and test {\iffieldundef{eprint}}}
    {\usebibmacro{url+urldate}}
    {}}
% Ende Literaturverzeichnis-Definitionen

% Start Kapitelseiten-Designs
\ifdesignEins
\addtokomafont{chapterprefix}{\raggedleft}
\renewcommand*{\chapterformat}{%
\mbox{\chapappifchapterprefix{\nobreakspace}%
\scalebox{3}{\color{chaptergrey}\thechapter\autodot}\enskip}}
\else
\fi
\newif\ifdesignEins
% zweites Design
\ifdesignZwei
\setkomafont{chapter}{\huge}
\setkomafont{chapterprefix}{\normalfont\normalsize\slshape}
\newkomafont{chapternumber}{\fontsize{100}{100}\selectfont}
\renewcommand\chapterformat{%
  \usekomafont{chapter}
  \raisebox{1.25\baselineskip}{\makebox[0pt][r]{\usekomafont{chapterprefix}\chapapp\enskip}}%
  \raisebox{-.5\baselineskip}{\usekomafont{chapternumber}\thechapter}%
}
\RedeclareSectionCommand[
  innerskip=1ex
]{chapter}
\newbox\chapternumberbox
\makeatletter
\renewcommand\chapterlinesformat[3]{%
  \Ifstr{#1}{chapter}{%
    \Ifstr{#2}{}{#3}{%
      \savebox\chapternumberbox{\chapterformat}%
      \parbox[t]{\dimexpr\textwidth-\wd\chapternumberbox-1em\relax}{\raggedchapter#3}%
       \quad#2%
    }%
    }{\@hangfrom{#2}{#3}}%
}
\makeatother
\renewcommand\raggedchapter{\raggedleft}
\else
\fi
\ifdesignDrei
\makeatletter
 \renewcommand*{\chapterformat}{%
   \begingroup
     \setlength{\unitlength}{1mm}%
     \begin{picture}(10,10)(0,5)
       \setlength{\fboxsep}{0pt}
       \put(10,15){\line(1,0){\dimexpr
           \textwidth-20\unitlength\relax\@gobble}}%
       \put(0,0){\makebox(10,20)[r]{%
           \fontsize{28\unitlength}{28\unitlength}\selectfont\thechapter
           \kern-.05em% Ziffer in der Zeichenzelle nach rechts schieben
         }}%
       \put(10,15){\makebox(\dimexpr
           \textwidth-20\unitlength\relax\@gobble,\ht\strutbox\@gobble)[l]{%
             \ \normalsize\color{black}\chapapp~\thechapter\autodot
           }}%
     \end{picture}
   \endgroup
}
\else
\fi
% Ende des Kapitelseiten-Designs

% L\"{o}st ein Problem, dass Abbildungen ans Ende des Kapitels schiebt
\renewcommand{\floatpagefraction}{0.8}
\renewcommand{\topfraction} {0.8}
\renewcommand{\bottomfraction} {0.5}
\renewcommand{\textfraction} {0.15}
\raggedbottom

% Neuberechnung des Satzspiegels nach Defintion aller Parameter
\recalctypearea



\pdfminorversion=7

% Wenn man die Kurzform der Journalnamen verwenden will, sind folgende Zeilen nützlich.
% z.B. einfach \actaa in shortjorunal einsetzen, erzeugt Acta.Astron.

\newcommand{\actaa}{Acta Astron.}   % Acta Astronomica
\newcommand{\araa}{Annu. Rev. Astron. Astrophys.}   % Annual Review of Astron and Astrophys
\newcommand{\aar}{Astron. Astrophys. Rev.} % Astrononmy and Astrophysics Review
\newcommand{\ab}{Astrobiol.}    % Astrobiology
\newcommand{\aj}{Astron. J.}   % Astronomical Journal
\newcommand{\apj}{Astrophys. J.}   % Astrophysical Journal
\newcommand{\apjl}{Astrophys. J. Lett.}   % Astrophysical Journal, Letters
\newcommand{\apjs}{Astrophys. J. Suppl. Ser.}   % Astrophysical Journal, Supplement
\newcommand{\ao}{Appl. Opt.}   % Applied Optics
\newcommand{\apss}{Astrophys. Space Sci.}   % Astrophysics and Space Science
\newcommand{\aap}{Astron. Astrophys.}   % Astronomy and Astrophysics
\newcommand{\aapr}{Astron. Astrophys. Rev.}   % Astronomy and Astrophysics Reviews
\newcommand{\aaps}{Astron. Astrophys. Suppl.}   % Astronomy and Astrophysics, Supplement
\newcommand{\baas}{Bull. Am. Astron. Soc.}   % Bulletin of the AAS
\newcommand{\caa}{Chinese Astron. Astrophys.}   % Chinese Astronomy and Astrophysics
\newcommand{\cjaa}{Chinese J. Astron. Astrophys.}   % Chinese Journal of Astronomy and Astrophysics
\newcommand{\cqg}{Class. Quantum Gravity}    % Classical and Quantum Gravity
\newcommand{\gal}{Galaxies}    % Galaxies
\newcommand{\gca}{Geochim. Cosmochim. Acta}   % Geochimica Cosmochimica Acta
\newcommand{\icarus}{Icarus}   % Icarus
\newcommand{\jcap}{J. Cosmol. Astropart. Phys.}   % Journal of Cosmology and Astroparticle Physics
\newcommand{\jgr}{J. Geophys. Res.}   % Journal of Geophysics Research
\newcommand{\jgrp}{J. Geophys. Res.: Planets}    % Journal of Geophysics Research: Planets
\newcommand{\jqsrt}{J. Quant. Spectrosc. Radiat. Transf.} % Journal of Quantitiative Spectroscopy and Radiative Transfer
\newcommand{\memsai}{Mem. Soc. Astron. Italiana}   % Mem. Societa Astronomica Italiana
\newcommand{\mnras}{Mon. Not. R. Astron. Soc.}   % Monthly Notices of the RAS
\newcommand{\nat}{Nature} % Nature
\newcommand{\nastro}{Nat. Astron.} % Nature Astronomy
\newcommand{\ncomms}{Nat. Commun.} % Nature Communications
\newcommand{\nphys}{Nat. Phys.} % Nature Physics
\newcommand{\na}{New Astron.}   % New Astronomy
\newcommand{\nar}{New Astron. Rev.}   % New Astronomy Review
\newcommand{\physrep}{Phys. Rep.}   % Physics Reports
\newcommand{\pra}{Phys. Rev. A}   % Physical Review A: General Physics
\newcommand{\prb}{Phys. Rev. B}   % Physical Review B: Solid State
\newcommand{\prc}{Phys. Rev. C}   % Physical Review C
\newcommand{\prd}{Phys. Rev. D}   % Physical Review D
\newcommand{\pre}{Phys. Rev. E}   % Physical Review E
\newcommand{\prl}{Phys. Rev. Lett.}   % Physical Review Letters
\newcommand{\psj}{Planet. Sci. J.}   % Planetary Science Journal
\newcommand{\planss}{Planet. Space Sci.}   % Planetary Space Science
\newcommand{\pnas}{Proc. Natl Acad. Sci. USA}   % Proceedings of the US National Academy of Sciences
\newcommand{\procspie}{Proc. SPIE}   % Proceedings of the SPIE
\newcommand{\pasa}{Publ. Astron. Soc. Aust.}   % Publications of the Astron. Soc. of Australia
\newcommand{\pasj}{Publ. Astron. Soc. Jpn}   % Publications of the Astron. Soc. of Japan (note no full stop following Jpn)
\newcommand{\pasp}{Publ. Astron. Soc. Pac.}   % Publications of the Astron. Soc. of the Pacific
\newcommand{\rsi}{Rev. Sci. Instrum.}   % Review of Scientific Instruments
\newcommand{\rmxaa}{Rev. Mexicana Astron. Astrofis.}   % Revista Mexicana de Astronomia y Astrofisica
\newcommand{\sci}{Science} % Science
\newcommand{\sciadv}{Sci. Adv.} % Science Advances
\newcommand{\solphys}{Sol. Phys.}   % Solar Physics
\newcommand{\sovast}{Soviet Ast.}   % Soviet Astronomy
\newcommand{\ssr}{Space Sci. Rev.}   % Space Science Reviews
\newcommand{\uni}{Universe} % Universe 